\Chapter{Background}
\Section{Problem setup}

Numerical integration problems are involved in fields such as physics, mathematical finance, biology and computer graphics. 
It usually happens when it is hard to solve some integral analytically. Therefore, one has to use numerical methods for such problems.  
The MC method is a general way to solve problems in such case\cite{fishman2013monte}. The method can be simply explained in the following way. 

Suppose we have the following standard integration approximation problem whose format is: 
\begin{equation}
    \label{eq:setup}
    I= \int_{[0,1)^d}f(\mathbf{x})\;\textrm{d}\mathbf{x}.
\end{equation}
Then we take a sample of $N$ IID points $\{\mathbf{X}_1,\dots, \mathbf{X}_N\}\in [0,1)^d$ following a uniform distribution, and construct the following MC estimator:
\[
    \hat{I}(f)=\frac{1}{N}\sum_{i=1}^{N}f(\mathbf{X_i}).
\]

However, there are several problems with the IID MC method\cite{niederreiter2010quasi}.
First, it is difficult to generate truly random samples. Second, an error bound for IID MC works only in the probabilistic sense. 
Last, in many applications the convergence rate of the MC method is considered not fast enough. 

Hence, the QMC method was introduced to address these problems. 
The QMC estimator is almost the same as MC estimator. 
The difference is that the sample points are taken from a low discrepancy sequence, which is deterministically chosen instead of random.   
We will briefly review one such sequence that we used for implementation of our method. 

\newpage

\Section{Sobol sequence}

Like we mentioned earlier, for QMC we do not use random points, so the question becomes how to choose such points?  
Naturally, in order to make a good approximation we want the error to go to $0$ as sample size $N$ increases, that is:
\[
    \lim_{N\to\infty}\frac{1}{N}\sum_{i=1}^{N}f(\mathbf{x_i})=\int_{[0,1)^d}f(\mathbf{x})\;\textrm{d}\mathbf{x}.
\]
The answer for how to choose such sample points leads to the theory of uniformly distributed sequences. 
One important property of such sequences is the ``fair'' interval \cite{dick2010digital}.
\theoremstyle{definition}
\begin{definition}{}
Let $\mathscr{P} =\{\mathbf{X}_1,\dots,\mathbf{X}_N\}$ be a finite point set in $[0, 1)^d$.
            For any given subset $J=[\mathbf{a},\mathbf{b})$ of $[0, 1)^d$, we say $J$ is $\textit{fair}$ if:  
    \[
        \frac{\sum_{i=1}^{N}1_{[\mathbf{a},\mathbf{b})}(\mathbf{X_i})}{N}
            =\prod_{j=1}^{d}(b_j-a_j)
                    \]
\end{definition}

This simply means that a measurable subset $J$ of a unit cube is fair if the number of sample points in it (left side of equation) is equal to its volume (right side of equation). 
Ideally, we want a sequence $\mathscr{P}$ that is fair for any given subset $J$. But such sequences do not exist. 
Hence, the condition for all intervals to be fair has to be weakened. This motivates the definition of $(t,d)$-sequence \cite{dick2010digital}. 
\theoremstyle{definition}
\begin{definition}{}
    For a given dimension $d \geq 1$, an integer base $b \geq 2$, a positive integer m, and an integer t with $0 \leq t < m$, a point set $\mathscr{P} =\{\mathbf{X}_1,\mathbf{X}_2,\dots\}$ in $[0, 1)^d$ is called a $\textit{$(t,d)$-sequence}$ in base $b$ 
if the point set $\{\mathbf{X}_{kb^m+1},\dots,\mathbf{X}_{kb^m+b^m}\}$ with all $m>t,\;k\geq 0$ is fair with respect to all intervals of the following form:
    \[
        J=\prod_{i=1}^{d}\Big[\frac{a_i}{b^{s_i}}, \frac{a_i+1}{b^{s_i}}\Big),
                    \]
                    where $s_i,\dots, s_d\in\mathbb{N}_0$ with $s_1+\dots+s_d=k$ and $0\leq a_i < b^{s_i}$  
\end{definition}

The Sobol sequence is the first constructed $(t,d)$-sequence in base 2 \cite{dick2010digital}. Another well known $(t,d)$-sequence is the Faure sequence, which is a $(0, d)$-sequences in a prime base $b \geq d$. 
The advantage of Sobol sequence is that it has a small base. 
Later in next chapter we will see this is a great help since the sample size takes the form of $b^m$. 
Also since we use computer for simulations, we gain some advantages on bit-level operations in base 2.
Figure~\ref{fg:sobol} shows a comparision between randomly generated sample points and Sobol sequence points in dimension 2. 
The latter looks more evenly scattered. 
In fact by the definition of $(t,d)$-sequence if we divide the unit square in to $256$ tiles evenly ($1\times256, 2\times128,\dots$), there will be exactly one point in each tile.  

\begin{figure}
    \centering
    \begin{subfigure}[b]{.5\textwidth}
    \includegraphics[width=\textwidth]{figures/cvSobolPlot1.eps}
    \caption{256 randomly generated points}
    \end{subfigure}~\begin{subfigure}[b]{.5\textwidth}
    \includegraphics[width=\textwidth]{figures/cvSobolPlot2.eps}
    \caption{256 Sobol points}
    \end{subfigure}
    \caption{Comparision between MC and QMC sample points}\label{fg:sobol}
\end{figure}

\Section{Control variates}

CV is a well known variance reduction technique used in MC simulation. 
It is ofen used when a ``simpler'' problem that is related to the original problem can be solved explicitly. In this section we briefly review the ideas and main results of the method.
 
Suppose we want to solve the integration problem~\eqref{eq:setup} showed earlier. Now suppose we know a function $h$ and the integral  
$\theta=\int_{[0,1)^d} h(\mathbf{x})\;\textrm{d}\mathbf{x}$. 
We then construct a new estimator:
\[\ICV(f)=\frac{1}{N}\sum_{i=1}^{N}\Big[ f(\mathbf{X}_i)-\BMC[h(\mathbf{X}_i)-\theta] \Big] \quad s.t.\; \mathbf{X}_i\sim \mathcal{U}[0,1)^d, \;\textrm{IID}.\]

We can easily see it's an unbiased estimator, i.e. $\mathbb{E}(\ICV) = I$.
Now the question is how should we choose $\BMC$ and why.
We know that the mean square error of the MC estimator is $\mathrm{Var}(\hat{I})+\mathrm{Bias}(\hat{I}^2)$. 
The CV method aims at efficiency improvment, so we need to reduce the mean square error. 
Since the estimator is unbiased, we only need to minimize its variance.
Hence, the optimal $\BMC$ should be the one that minmize the variance of esimator.
Here we give a simple derivation of optimal $\BMC$ for single CV.
First, the variance of $\ICV$ is: 
\begin{align*}
	\mathrm{Var}(\ICV)
    =&\mathrm{Var}\Big( \frac{1}{N}\sum_{i=1}^{N}\big[ f(\mathbf{X}_i)-\BMC[h(\mathbf{X}_i)-\theta] \big]\Big)\\
    =&\frac{1}{N}\mathrm{Var}\Big(f(\mathbf{X}_i)-\BMC[h(\mathbf{X}_i)-\theta]\Big)\quad \text{by $\mathrm{X}_i$ are IID} \\
    =&\frac{1}{N}\mathbb{E}\Big(\big[f(\mathbf{X}_i)-\BMC[h(\mathbf{X}_i)-\theta]-I\big]^2 \Big) \\
    =&\frac{1}{N}\mathbb{E}\Big(\big[ [f(\mathbf{X}_i)-I] -\BMC[h(\mathbf{X}_i)-\theta]\big]^2 \Big) \\
    =&\frac{1}{N}\mathbb{E}\big([f(\mathbf{X}_i)-I]^2-2\BMC[f(X_i)-I][h(\mathbf{X}_i)-\theta]+\BMC^2[h(\mathbf{X}_i)-\theta]^2 \Big)\\
    =&\frac{1}{N}\Big(\mathrm{Var}[f(\mathbf{X}_i)]-2\BMC\mathrm{Cov}[f(X_i),h(\mathbf{X}_i)]+\BMC^2\mathrm{Var}[h(\mathbf{X}_i)]\Big)\\
    =&\frac{1}{N}\Big(\mathrm{Var}[h(\mathbf{X}_i)]\big(\BMC-\frac{\mathrm{Cov}[f(X_i),h(\mathbf{X}_i)]}{\mathrm{Var}[h(\mathbf{X}_i)]}\big)^2+\\
     &\quad \quad \mathrm{Var}[f(\mathbf{X}_i]-\frac{\mathrm{Cov}^2[f(\mathbf{X}_i),h(\mathbf{X}_i)]}{\mathrm{Var}[h(\mathbf{X}_i)]} \Big),
\end{align*}
then the optimal $\BMC$ is given by: 
\begin{equation}
    \BMC^*=\frac{\mathrm{Cov}[f(\mathbf{X}_i),h(\mathbf{X}_i)]}{\mathrm{Var}[h(\mathbf{X}_i)]}.
    \label{eq:optBeta}
\end{equation}
Note that in practice since $\mathrm{Cov}[f(\mathbf{X}),h(\mathbf{X})]$ or $\mathrm{Var}[h(\mathbf{X})]$ is unknown we use estimations to compute $\BMC^*$.  
In this case the variance become:
\[
    \mathrm{Var}(\ICV)= \frac{\mathrm{Var}[f(\mathbf{X}_i)]}{N}\big(1-\mathrm{corr}^2[f(\mathbf{X}_i), h(\mathbf{X}_i)]\big),
\]
and note we always have: 
\[
\mathrm{Var}(\ICV) \leq \frac{\mathrm{Var}[f(\mathbf{X}_i)]}{N}=\mathrm{Var}(\hat{I}).
\]

Now we can see the merit of control variates as a variance reduction method. 
In the worst case, we get a completely uncorrelated $h$ that leads correlation to zero, and we have variance exactly the same as not using control variates. On the other hand, the more correlated our control variates is to the target function, the more variance we can get rid of by using the method.

\Section{Reliable adaptive QMC with digital sequence}

One practical problem for QMC method is that how to get the sample size big enough for a required error tolerance. 
The idea in work of Hickernell and Jim{\'e}nez Rugama (2014)\cite{hickernell2014reliable} is to construct a QMC algorithm with reliable error estimation using digital sequences. 
Here we briefly summarize their results.

The error of QMC method on digital sequence can be expressed in terms of Walsh coefficients of the integrand on certain cone conditions. 
\begin{align}
    \label{eq:errBound}
    &\text{ if } f \in \mathscr{C}\text{ then } \Big|\int_{[0,1)^d}f(\mathbf{x})\;\textrm{d}\mathbf{x} - \hat{I}_m(f)\Big| \leq a(r,m) \sum_{\lfloor b^{m-r-1} \rfloor}^{b^{m-r}-1} |\tilde{f}_{m,\kappa}|\\
    &\hat{I}_m(f): = \frac{1}{b^m}\sum_{i=0}^{b^m-1}f(\mathbf{X}_i)
    \text{ ,$\;\mathbf{X}_i\in$ digital sequence}\notag\\
    &\tilde{f}_{m,\kappa}=\text{ discrete Walsh coefficients of }f \notag\\
    &a(r,m) =\text{ inflation factor that depends on } \mathscr{C} \notag.
\end{align}
Here is the definition of the cone condition:
\begin{align}
   &\mathscr{C}:=\Big\{f\in L^2[0,1)^d:\;\bigcirc \leq \hat{\omega}(m-l)\Diamond,\; l\leq m;\quad
   \Diamond \leq \mathring{\omega}(m-l) \Box, 
   \; l^*\leq l \leq m\Big\}\notag\\
   \label{eq:cone}
   &\bigcirc:= \sum_{\kappa=\lfloor b^{l-1} \rfloor}^{b^l-1} \sum_{\lambda=1}^{\infty}| \hat{f}_{\kappa+\lambda b^m}|,\quad  
   \Diamond:=\sum_{\kappa=b^m}^{\infty}|\hat{f}_{\kappa}|,\quad
   \Box:= \sum_{\kappa=b^{l-1}}^{b^l-1}|\hat{f}_\kappa|\\
   &l^*\in \mathbb{N}\text{ be fixed }; \forall m\in \mathbb{N},\hat{\omega}(m),\mathring{\omega}(m)\geq 0, \text{ and } \lim_{m\to \infty} \hat{\omega}(m)=0,\; \lim_{m\to \infty} \mathring{\omega}(m)=0\notag.
\end{align}

The first inequality ($\bigcirc \leq \Diamond$) tries to bound the partial sum of Walsh coefficients with the entire sum of the same Walsh coefficients. 
Take $b=2,\; l^*=0,\; m=12$ for example. 
In figure~\ref{fg:cone} the sum of $\bigcirc$ should be bounded by some inflation factor times the sum of $\Diamond$. 
Now since we still can not estimate sum of Walsh coefficients in the form of $\Diamond$, the second inequality ($\Diamond\leq \Box$) tries to bound the sum of the larger indexed Walsh coefficients by the sum of smaller indexed Walsh coefficients. 
Take $l=8$ at this time, which means in Figure~\ref{fg:cone} the sum of $\Diamond$ should be bounded by some inflation factor times the sum of $\Box$.

The definition of the cone provides some constraints on the functions about how they should behave in order to use the error bound formula~\eqref{eq:errBound}. 
This means that $|\hat{f}_\kappa|$ does not dramatically bounce back as $\kappa$ goes to infinity. 
Note that in Figure~\ref{fg:cone} we call $\bigcirc$ the error bound, this is proven to be true and under the cone conditions we can estimate it using discrete Walsh coefficients instead of true Walsh coefficients.
\begin{figure}[h]
    \centering
    \includegraphics[width=\textwidth]{figures/cone.eps}
    \caption{Cone condition for reliable adpative QMC algorithm}
    \label{fg:cone}
\end{figure}
