\Chapter{Background}
\Section{Problem Setup}

Numerical integration problems is involved in fields such as physics,
, mathematical finance, biology, computer graphics, and many others fields.It usually happens when it is hard to solve some integral analytically. Therefore one has to use numerical methods for such problems. 
And Monte-Carlo method is the general way to solve problems in such case\cite{fishman2013monte}. The method can be simply explained in the following way. 

Suppose we have the following standard integration approximation problem whose format is 
\begin{equation}
    \label{eq:setup}
    I= \int_{[0,1)^d}f(x)dx
\end{equation}

Then we will take sample of $n$ points $\{\mathbf{x_0},\dots, \mathbf{x_n}\}\in [0,1)^d$ follow the uniform distribuation randomly, then form a estimator in form of  
\[
\hat{I}(f)=\frac{1}{n}\sum_{i=1}^{n}f(X_i)
\]

However, there are several problem with simple Monte-Carlo method\cite{niederreiter2010quasi}.
First, it is difficult to generate truly random samples. Second, error bound for MC works only probabilistic sense. And last, in many applications the
convergence rate of MC is considered not fast enough. 

Hence QMC method were introduced to address these problems. For QMC method the estimator is almost the same with MC. The difference is that the sample points are taken from low discrepency sequence, which is determinsticlly chosen instead of random.   
We will briefly rewiew one method for constructing such sequence that we used for our application in the next section. 

\Section{Sobol Sequence}


\Section{Control Variates}

Control variates has been know as variance reduction technique used in Monte Carlo methods. In this section we will brief review the ideas and some results of this methods.
 
Suppose we want to solve the integration problem~\eqref{eq:setup} showed earlier, now we have a known function $h$ and its value on the interval
$\int_{[0,1)^d} h(x)dx = \theta$.  We then construct a new estimator as the following 
\[ \hat{I}_{cv}(f)=\frac{1}{n}\sum_{i=1}^{n}\Big[ f(X_i)-\beta_{mc}[h(X_i)-\theta] \Big] \quad s.t.\; X_i\sim \mathcal{U}[0,1), \; i.i.d.\]

We can easily see it's an unbiased estimator, i.e. $\mathbb{E}(\hat{I}_{cv}) = I$.
Now we want to pick the right $\beta_{mc}$ such that make the estimation more efficient.
For simplicity, we consider single control variate in this case.
Base on previous MC error estimating formula~\eqref{}, we know it's achievable by minimizing the variance of the estimator, which is 
\begin{align*}
	\mathrm{Var}_{mc}(\hat{I}_{cv})
    =&\mathrm{Var}\Big( \frac{1}{n}\sum_{i=1}^{n}\big[ f(X_i)-\beta_{mc}[h(X_i)-\theta] \big]\Big)\\
    =&\frac{1}{n}\mathrm{Var}\Big(f(X_i)-\beta_{mc}[h(X_i)-\theta]\Big)\quad \text{by $X_i$ i.i.d} \\
    =&\frac{1}{n}\mathbb{E}\Big(\big[f(X_i)-\beta_{mc}[h(X_i)-\theta]-I\big]^2 \Big) \\
    =&\frac{1}{n}\mathbb{E}\Big(\big[ [f(X_i)-I] -\beta_{mc}[h(X_i)-\theta]\big]^2 \Big) \\
    =&\frac{1}{n}\mathbb{E}\big([f(X_i)-I]^2-2\beta_{mc}[f(X_i)-I][h(X_i)-\theta]+\beta_{mc}^2[h(X_i)-\theta]^2 \Big)\\
    =&\frac{1}{n}\Big(\mathrm{Var}[f(X_i)]-2\beta_{mc}\mathrm{Cov}[f(X_i),h(X_i)]+\beta_{mc}^2\mathrm{Var}[h(X_i)]\Big)\\
=&\frac{1}{n}\Big(\mathrm{Var}[h(X_i)](\beta_{mc}-\frac{\mathrm{Cov}[f(X_i),h(X_i)]}{\mathrm{Var}[h(X_i)]})^2+\mathrm{Var}[f(X_i]-\frac{\mathrm{Cov}^2[f(X_i),h(X_i)]}{\mathrm{Var}[h(X_i)]} \Big)
\end{align*}

Then the optimal $\beta_{mc}$ is given by 
\begin{equation}
    \beta_{mc}^*=\frac{\mathrm{Cov}[f(X_i),h(X_i)]}{\mathrm{Var}[h(X_i)]}
    \label{eq:optBeta}
\end{equation}

In which case the variance become
\[
    \mathrm{Var}_{mc}(\hat{I}_{cv})= \frac{\mathrm{Var}[f(X_i)]}{n}\big(1-\mathrm{corr}^2[f(X_i), h(X_i)]\big) 
\]

and we always have 
\[
\mathrm{Var}_{mc}(\hat{I}_{cv}) \leq \frac{\mathrm{Var}[f(X_i)]}{n}=\mathrm{Var}_{mc}(\hat{I})
\]

Now we can see the merit of control variates as a variance reduction method.In worst case, we get a completely uncorrelated $g$ that leads correlation to zero, and we have variance exactly the same as not using control variates. On the other hand, the more correlated our control variates is to the target function, the more variance we can get rid of by using the method.

\Section{Reliable Adaptive QMC with digital sequence}

\Subsection{Idea of adaptive cubature algorithm}

One practical problem for QMC method is that how to get the sample size big enough for a required error tolerance. The idea in work of Hickernell and Jiménez Rugama(2014)\cite{hickernell2014reliable} is to construct a QMC algorithm with reliable error estimation on digital sequence. Here we briefly summarize their results.

The error of QMC method on digital sequence can be expressed in terms of Walsh coefficients of the integrand on certain cone conditions. 
\begin{align}
    \label{eq:errBound}
    &\text{ if } f \in \mathscr{C}\text{ then } \Big|\int_{[0,1)^d}f(x)dx - \hat{I}_m(f)\Big| \leq a(r,m) \sum_{\lfloor 2^{m-r-1} \rfloor}^{2^{m-r}-1} |\tilde{f}_{m,k}|\\
    &\hat{I}_m(f): = \frac{1}{b^m}\sum_{i=0}^{b^m-1}f(z_i\oplus \Delta)\notag\\
    &\tilde{f}_{m,k}=\text{ discrete Walsh coefficients of }f \notag\\
    &a(r,m) =\text{ inflation factor that depends on } \mathcal{C} \notag
\end{align}

Here is the defination of the cone condition.
\begin{align}
   &\mathscr{C}:=\Big\{f\in L^2[0,1)^d:\;\bigcirc \leq \hat{\omega}(m-l)\Diamond,\; l\leq m;\quad
   \Diamond \leq \mathring{\omega}(m-l) \Box, 
   \; l^*\leq l \leq m\Big\}\notag\\
   \label{eq:cone}
   &\bigcirc:= \sum_{\kappa=\lfloor b^{l-1} \rfloor}^{b^l-1} \sum_{\lambda=1}^{\infty}| \hat{f}_{\kappa+\lambda b^m}|,\quad  
   \Box:= \sum_{\kappa=b^{l-1}}^{b^l-1}|\hat{f}_\kappa|,\quad
   \Diamond:=\sum_{\kappa=b^m}^{\infty}|\hat{f}_{\kappa}|\\
   &l^*\in \mathbb{N}\text{ be fixed }; \forall m\in \mathbb{N},\hat{\omega}(m),\mathring{\omega}(m)\geq 0, \text{ and } \lim_{m\to \infty} \hat{\omega}(m)=0,\; \lim_{m\to \infty} \mathring{\omega}(m)=0\notag
\end{align}

The first inequality($\bigcirc \leq \Diamond$) means the sum of the larger indexed Walsh coefficients bounds a partial sum of the same coeffcients. Take $l=0, m=12$ for example, in Figure~\ref{fg:cone} the the sum of circles should be bounded by some factor times the sum of diamonds. The second inequality($\Diamond\leq \Box$) requires the sum of the larger Walsh coefficients be bounded by the sum of smaller indexed Walsh coefficients. Take $l=8$ at this time, which means in Figure~\ref{fg:cone} the sum of diamonds should be bounded by some relax factor times the squares.

The cone give some meanings for the functions about how they should behave to get the err bound formula~\eqref{eq:errBound}. This means that $|\hat{f}_\kappa|$ does not dramatically bounce back as $\kappa$ goes to infinity. Note that in Figure~\ref{fg:cone} we call circles the err bound, this is proven to be true and under the cone conditions we can estimate it using discrete Walsh coefficients instead of true Walsh coefficients.
\begin{figure}[h]
    \centering
    \includegraphics[width=\textwidth]{figures/cone.eps}
    \caption{Cone condition for reliable adpative QMC algorithm}
    \label{fg:cone}
\end{figure}
