\Chapter{Background}

\section{QMC}

\subsection{Digital Sequence}
Talk about the whole idea briefly.
\subsection{QMC}
Introduce QMC.


\section{Control Variates}

\subsection{A Brief Review}
Control variates has been know as variance reduction technique used in Monte Carlo methods.In this part we will brief review some crucial idea of this methods so we can see what's the problem for using it with QMC. \\ 
Suppose we have the following integration approximation problem:
\[I= \int_{[0,1]^d}f(x)dx\]
If we use Monte-Carlo method, the estimator should be: 
\[
\hat{I}(f)=\frac{1}{n}\sum_{i=1}^{n}f(X_i), X_i\sim \mathcal{U}[0,1)^d
\]
Now suppose we have a known function $h$ and its value on the interval
$\int h(x)dx = \theta$.
We construct a new estimator as the following: 
\[ \hat{I}_{cv}(f)=\frac{1}{n}\sum_{i=1}^{n}\Big( f(X_i)-\beta_{mc}(g(X_i)-\theta) \Big) \quad s.t. X_i\sim \mathcal{U}[0,1)\]
We can easily see it's an unbiased estimator.($\mathbb{E}(\hat{I}_{cv}) = I$)\\
Now we want to pick the right $\beta_{mc}$ such that make the estimation more efficient. Base on previous MC error estimating formula~\eqref{}, we know its achievable by minimizing the variance of the estimator, which is: 
\begin{align*}
	\quad &\mathrm{Var}_{mc}(\hat{I}_{cv}) \\
	=& \frac{\mathbb{E}\big([f(X_i)-\beta_{mc}(g(X_i)-\theta)-I]^2 \big)}{n} \\
	=& \frac{\mathbb{E}\big([(f(X_i)-I)^2-2\beta_{mc}(f(X_i)-I)(g(X_i)-\theta)-\beta_{mc}^2(g(X_i)-\theta)^2] \big)}{n}\\
	=& \frac{\mathrm{Var}(f(X_i)-2\beta_{mc}\mathrm{Cov}(f(X_i),g(X_i))+\beta_{mc}^2\mathrm{Var}(g(X_i)) }{n}\\
	=& \frac{\mathrm{Var}(g(X_i)(\beta_{mc}-\frac{\mathrm{Cov}(f(X_i),g(X_i))}{\mathrm{Var}(g(X_i))})^2+\mathrm{Var}(f(X_i)-\frac{\mathrm{Cov}^2(f(X_i),g(X_i))}{\mathrm{Var}(g(X_i))} }{n}
\end{align*}
then the optimal $\beta_{mc}$ is given by: 
$\beta_{mc}=\frac{\mathrm{Cov}(f(X_i),g(X_i))}{\mathrm{Var}(g(X_i))}$\\
In which case the variance become:
\[
\mathrm{Var}_{mc}(\hat{I}_{cv})= \frac{\mathrm{Var}(f(X_i)}{n}[1-\mathrm{corr}^2(f(X_i), g(X_i))]
\]

\subsection{Control Variates with QMC}
Suppose $X_1, \dots, X_n$ are generated by QMC rule, the estimator stays the same.
We can prove it is still unbiased:
\[
\mathbb{E}(\hat{I}_{cv})=\mathbb{E}\Big(\frac{1}{n}\sum_{i=1}^{n}\Big( f(X_i)-\beta(g(X_i)-\theta)\Big)=I 
\]
However, it's not the same case for $\beta_{rqmc}$ since we do not have I.I.D. for $X_i$ this time
\begin{align*}
\mathrm{Var}_{rqmc}(\hat{I}_{cv}) &\not= \frac{\mathbb{E}\big([f(X_i)-\beta_{mc}(g(X_i)-\theta)-I]^2 \big)}{n}\\
&=\mathrm{Var}_{rqmc}\Big( \hat{I}- \beta_{rqmc}\hat{G}\Big)\quad , \hat{G}=\sum_{i=1}^{n}(g(X_i)-\theta)\\
\beta_{rqmc}^*&= \mathrm{Var} (\hat{G})^{-1}\mathrm{Cov} (\hat{G}, \hat{I})
\end{align*}

\section{Reliable Adaptive QMC with digital sequence}
\subsection{Idea of adaptive cubature algorithm}
One practical problem for QMC method is that how to get the sample size big enough for a required error tolerance. 
The idea in work of Hickernell and Jiménez Rugama(2014) is to construct a 
QMC algorithm with reliable error estimation on digital sequence. 
Here we briefly summarize their results.
The error of QMC method on digital sequence can be expressed in terms of Walsh coefficients of the integrand on certain cone conditions. 
\begin{align*}
&\Big|\int_{[0,1)^d}f(x)dx - \hat{I}_m(f)\Big| \leq a(r,m) \sum_{\lfloor 2^{m-r-1} \rfloor}^{2^{m-r}-1} |\hat{f}_{m,k}|\\
&\hat{I}_m(f): = \frac{1}{b^m}\sum_{i=0}^{b^m-1}f(z_i\oplus \Delta)\\
&\hat{f}_{m,k}=\text{ discrete Fourier coefficients of }f\\
&a(r,m) =\text{ inflation factor that depends on } \mathcal{C}
\end{align*}
Here is the cone condition.
\begin{align*}
	&\Big|\int_{[0,1)^d}f(x)dx - \hat{I}_m(f)\Big|
    \leq \sum {\bigcirc} 
	\leq \sum {\Box}
	\leq a(r,m) \sum_{\lfloor b^{m-r-1} \rfloor}^{b^{m-r}-1}|\hat{f}_{m,k}|\\
    &\bigcirc:= \sum_{\lambda=1}^{\infty}| \hat{f}_{\lambda b^m}|,\quad  
    \Box:= \sum_{\kappa=b^{l-1}}^{b^l-1}|\hat{f}_\kappa|,\quad
    \Diamond:=\sum_{\kappa=b^m}^{\infty}|\hat{f}_{\kappa}|\\
    &\mathcal{C}:=\Big\{\sum{\bigcirc} \leq \sum{\Diamond} \leq \sum{\Box}\Big\}
\end{align*}

\includegraphics[width=0.8\textwidth]{figures/cone.bmp}
