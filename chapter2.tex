\Chapter{Background}
\Section{Problem setup}

Numerical integration problems are involved in fields such as physics, mathematical finance, biology, computer graphics, and many others fields. 
It usually happens when it is hard to solve some integral analytically. Therefore, one has to use numerical methods for such problems.  
MC method is the general way to solve problems in such case\cite{fishman2013monte}. The method can be simply explained in the following way. 

Suppose we have the following standard integration approximation problem whose format is: 
\begin{equation}
    \label{eq:setup}
    I= \int_{[0,1)^d}f(x)dx.
\end{equation}
Then we take sample of $n$ points $\{\mathbf{x_0},\dots, \mathbf{x_n}\}\in [0,1)^d$ follow the uniform distribuation randomly, and construct the following MC estimator:
\[
\hat{I}(f)=\frac{1}{n}\sum_{i=1}^{n}f(X_i).
\]

However, there are several problems with IID MC method\cite{niederreiter2010quasi}.
First, it is difficult to generate truly random samples. Second, error bound for IID MC works only in probabilistic sense. 
Second, in many applications the convergence rate of MC is considered not fast enough. 

Hence, QMC method were introduced to address these problems. 
For QMC method the estimator is almost the same with MC. 
The difference is that the sample points are taken from low discrepency sequence, which is determinsticlly chosen instead of random.   
We will briefly rewiew one realization such sequence that we used for implementation of our method. 

\newpage

\Section{Sobol sequence}

First is the definition for discrepancy \cite{dick2010digital}.
Let $\mathscr{P} =\{x_0,\dots,x_{N −1}\}$ be a finite point set in $[0, 1)^s$.
The extreme discrepancy $D_N$ of this point set is defined as:
\[
    D_N(\mathscr{P}) := \sup_{J}\Big| \frac{A(J, N)}{N}-\lambda_s(J)\Big|.
\]
where the supremum is extended over all sub-intervals $J \subseteq [0, 1)^s$ of the form $J=[\mathbf{a},\mathbf{b})$.
It is a quantitative measure for the deviation of a finite point set from uniform distribution.
The worst-case discrepancy is called the star-discrepancy and is defined as:
\begin{align*}
    D^*_N(\mathscr{P})&:=\sup_{x\in[0,1)^d}|\Delta_{\mathscr{P}}(x)|\\
    \Delta_{\mathscr{P}}(x)&:=A([0,x),N,P)/N - \lambda_s([0,x)).
\end{align*}
The goal of a low-discrepancy sequence is to minimize this star discrepancy.
Initially low-discrepancy sequences were not designed with digital arithmetic in mind. 
The Van der Corput sequence \cite{halton1960efficiency} is an example of such sequences.
Modern low-discrepancy sequences are called digital sequences \cite{l2005recent}. This is essentially because they are constructed using binary operations and are therefore well-suited to efficient implementations on computers.
Sobol sequence was the first constructed digital sequences in base 2 in 1967 \cite{dick2010digital}.

\Section{Control variates}

CV is a well known variance reduction technique used in MC simulation. 
It is ofen used when a 'simpler' version of the origin problem can be solved explicitly. In this section we briefly review the ideas and main results of the method.
 
Suppose we want to solve the integration problem~\eqref{eq:setup} showed earlier, now we have a known function $h$ and its value on the interval
$\int_{[0,1)^d} h(x)dx = \theta$. 
We then construct a new estimator as the following:
\[\ICV(f)=\frac{1}{n}\sum_{i=1}^{n}\Big[ f(X_i)-\BMC[h(X_i)-\theta] \Big] \quad s.t.\; X_i\sim \mathcal{U}[0,1), \; i.i.d.\]

We can easily see it's an unbiased estimator, i.e. $\mathbb{E}(\ICV) = I$.
Now the question is how should we give $\BMC$ and why is that.
The idea is rather straitforward. 
We know the mean square error of MC estimator is $\mathrm{Var}(\hat{I})+\mathrm{Bias}(\hat{I}^2$\cite{}. 
CV method aims at efficiency improvment, so we need to reduce mean square error. 
Since the estimator is unbiased, we only need to minimize its variance.
Hence, the optimal $\BMC$ should be the one that minmize the variance of esimator.
Here we give a simple derivation of optimal $\BMC$ for single CV.
First, the variance of $\ICV$ is: 
\begin{align*}
	\mathrm{Var}(\ICV)
    =&\mathrm{Var}\Big( \frac{1}{n}\sum_{i=1}^{n}\big[ f(X_i)-\BMC[h(X_i)-\theta] \big]\Big)\\
    =&\frac{1}{n}\mathrm{Var}\Big(f(X_i)-\BMC[h(X_i)-\theta]\Big)\quad \text{by $X_i$ i.i.d} \\
    =&\frac{1}{n}\mathbb{E}\Big(\big[f(X_i)-\BMC[h(X_i)-\theta]-I\big]^2 \Big) \\
    =&\frac{1}{n}\mathbb{E}\Big(\big[ [f(X_i)-I] -\BMC[h(X_i)-\theta]\big]^2 \Big) \\
    =&\frac{1}{n}\mathbb{E}\big([f(X_i)-I]^2-2\BMC[f(X_i)-I][h(X_i)-\theta]+\BMC^2[h(X_i)-\theta]^2 \Big)\\
    =&\frac{1}{n}\Big(\mathrm{Var}[f(X_i)]-2\BMC\mathrm{Cov}[f(X_i),h(X_i)]+\BMC^2\mathrm{Var}[h(X_i)]\Big)\\
    =&\frac{1}{n}\Big(\mathrm{Var}[h(X_i)](\BMC-\frac{\mathrm{Cov}[f(X_i),h(X_i)]}{\mathrm{Var}[h(X_i)]})^2+\\
    &\quad \quad \mathrm{Var}[f(X_i]-\frac{\mathrm{Cov}^2[f(X_i),h(X_i)]}{\mathrm{Var}[h(X_i)]} \Big),
\end{align*}
then the optimal $\BMC$ is given by: 
\begin{equation}
    \BMC^*=\frac{\mathrm{Cov}[f(X_i),h(X_i)]}{\mathrm{Var}[h(X_i)]}.
    \label{eq:optBeta}
\end{equation}
In this case the variance become:
\[
    \mathrm{Var}(\ICV)= \frac{\mathrm{Var}[f(X_i)]}{n}\big(1-\mathrm{corr}^2[f(X_i), h(X_i)]\big),
\]
and note we always have: 
\[
\mathrm{Var}(\ICV) \leq \frac{\mathrm{Var}[f(X_i)]}{n}=\mathrm{Var}(\hat{I}).
\]

Now we can see the merit of control variates as a variance reduction method. 
In the worst case, we get a completely uncorrelated $g$ that leads correlation to zero, and we have variance exactly the same as not using control variates. On the other hand, the more correlated our control variates is to the target function, the more variance we can get rid of by using the method.

\Section{Reliable adaptive QMC with digital sequence}

\Subsection{Idea of adaptive cubature algorithm}

One practical problem for QMC method is that how to get the sample size big enough for a required error tolerance. The idea in work of Hickernell and Jiménez Rugama(2014)\cite{hickernell2014reliable} is to construct a QMC algorithm with reliable error estimation on digital sequence. Here we briefly summarize their results.

The error of QMC method on digital sequence can be expressed in terms of Walsh coefficients of the integrand on certain cone conditions. 
\begin{align}
    \label{eq:errBound}
    &\text{ if } f \in \mathscr{C}\text{ then } \Big|\int_{[0,1)^d}f(x)dx - \hat{I}_m(f)\Big| \leq a(r,m) \sum_{\lfloor 2^{m-r-1} \rfloor}^{2^{m-r}-1} |\tilde{f}_{m,k}|\\
    &\hat{I}_m(f): = \frac{1}{b^m}\sum_{i=0}^{b^m-1}f(z_i\oplus \Delta)\notag\\
    &\tilde{f}_{m,k}=\text{ discrete Walsh coefficients of }f \notag\\
    &a(r,m) =\text{ inflation factor that depends on } \mathscr{C} \notag.
\end{align}
Here is the defination of the cone condition:
\begin{align}
   &\mathscr{C}:=\Big\{f\in L^2[0,1)^d:\;\bigcirc \leq \hat{\omega}(m-l)\Diamond,\; l\leq m;\quad
   \Diamond \leq \mathring{\omega}(m-l) \Box, 
   \; l^*\leq l \leq m\Big\}\notag\\
   \label{eq:cone}
   &\bigcirc:= \sum_{\kappa=\lfloor b^{l-1} \rfloor}^{b^l-1} \sum_{\lambda=1}^{\infty}| \hat{f}_{\kappa+\lambda b^m}|,\quad  
   \Box:= \sum_{\kappa=b^{l-1}}^{b^l-1}|\hat{f}_\kappa|,\quad
   \Diamond:=\sum_{\kappa=b^m}^{\infty}|\hat{f}_{\kappa}|\\
   &l^*\in \mathbb{N}\text{ be fixed }; \forall m\in \mathbb{N},\hat{\omega}(m),\mathring{\omega}(m)\geq 0, \text{ and } \lim_{m\to \infty} \hat{\omega}(m)=0,\; \lim_{m\to \infty} \mathring{\omega}(m)=0\notag.
\end{align}

The first inequality($\bigcirc \leq \Diamond$) means the sum of the larger indexed Walsh coefficients bounds a partial sum of the same coeffcients. 
Take $l=0, m=12$ for example, in Figure~\ref{fg:cone} the the sum of circles should be bounded by some factor times the sum of diamonds. The second inequality($\Diamond\leq \Box$) requires the sum of the larger Walsh coefficients be bounded by the sum of smaller indexed Walsh coefficients. 
Take $l=8$ at this time, which means in Figure~\ref{fg:cone} the sum of diamonds should be bounded by some relax factor times the squares.

The cone give some meanings for the functions about how they should behave to get the err bound formula~\eqref{eq:errBound}. This means that $|\hat{f}_\kappa|$ does not dramatically bounce back as $\kappa$ goes to infinity. Note that in Figure~\ref{fg:cone} we call circles the err bound, this is proven to be true and under the cone conditions we can estimate it using discrete Walsh coefficients instead of true Walsh coefficients.
\begin{figure}[h]
    \centering
    \includegraphics[width=\textwidth]{figures/cone.eps}
    \caption{Cone condition for reliable adpative QMC algorithm}
    \label{fg:cone}
\end{figure}
