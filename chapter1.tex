\Chapter{Introduction}

\Section{The idea}

Recently there are some great results from construction of Quasi-Monte Carlo (QMC) methods that can adaptively choose a sample size for given error tolerances\cite{hickernell2014reliable}.   
Our work try to combine the reliable adaptive QMC (RAQMC) methods with control variates (CV). We will justify the theory behind it, construct a practical algorithm which can be implemented and tested through high dimensional integration examples.
CV is a variance reduction technique for independent and identically distributed (i.i.d.) Monte Carlo (MC) methods.
QMC can be viewed as a deterministic version of IID MC, which outperforms MC for many integrals\cite{avramidis1996integrated}. 
Naturally we wonder if QMC can also benefit from the CV technique. If that is possible, it can be especially useful for problems where we can easily find good CV.

\Section{The challenge}

The challenge is that for QMC the quadrature points are deterministic instead of random, so the variance minization can not be used. 
Even if one tries to use the randomized QMC, the optimal control variate coefficient for QMC is generally not the same as for IID MC as explained by Hickernell, Lemieux, and Owen\cite{hickernell2005control}. 
This requires us to figure out a new way to get the optimal coefficients for CV with QMC.

\Section{Outline}

In chapter 2 we first briefly talk about QMC rules and their difference from IID MC. 
Then we introduce CV and the RAQMC algorithm. 
In Chapter 3 we derive our method and provide its justification.
In chapter 4 we demonstrate results from several numerical experiments. For the final chapter we conclude the results and discuss unsolved problems as well as possible improvemnts to our method.
