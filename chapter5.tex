\Chapter{Conclusion}
\Section{Discussion}

So far there are only few QMC algorithms that can adaptively determine the sample size needed based on integrand values. 
This is because the estimation of error for QMC is hard. 
Several studies show that if using quasi standard error there will be some serious drawbacks\cite{owen2006warnock}. 
There is also a way using internal replications of i.i.d. Randomized QMC rules, but the number of replications are not known\cite{hickernell2005control}.

For control variates with QMC the research progress is also limited since its hard to estimate the value of $\beta$ as we stated in chapter 3. 
Hickernell and Llu{\'\i}s (2014) \cite{hickernell2014reliable}'s work on building a QMC method provided a reliable and adaptive way to use QMC, as well as 
gave us insights into combining reliable QMC with control variates.    
The main idea of this reliable QMC is to bound the error of estimation using summation of part Walsh coefficients. 
We utilized the same idea for calculating the optimal coefficient for control variates.  
In order to compensate the extra computation cost for control variates, we used several technique in our design to keep those cost minimal.

We tested our algorithm on several option pricing problem under Black-Scholes scheme. We found the accuracy of algorithm is consistent and reliable. 
Comparison of QMC with CV and normal QMC is performed on different pairs of options, the results show that using the proper control variates can have great improvement over normal QMC methods.     

\Section{Future work}

Currently we only have the method implemented in sobol sequences. 
One further possible work is that we can extend this method on rank-1 lattices\cite{rugama2014adaptive}. The idea for getting CV coefficients is the same, but due to the different structure for digital sequences, some effort has to be done for adaption of the method.
